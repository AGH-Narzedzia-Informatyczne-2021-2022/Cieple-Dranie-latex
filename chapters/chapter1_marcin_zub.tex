\label{sec:marcinzub}

\section{Indeksy giełdowe}

Często spotykam się ze stwierdzeniem, że \textit{"inwestowanie jest jak hazard"} - nie można być pewnym swoich decyzji, a większość to w ogóle zależy od szczęścia. To niestety bardzo mylne podejście, które zdaje się odstraszać wielu potencjalnych inwestorów. W dobie nieustannie rosnącej inflacji, co raz to większych podatków i trudnej sytuacji na rynku pracy, jakakolwiek forma stabilizacji finansowej zdaje się być wręcz niemożliwa do uzyskania. Ale czy aby napewno? 

\noindent Artykuł ma na celu przedstawić najpopularniejsze indeksy giełdowe, przedstawić ich zmienność i zachęcić do zapoznania się z ideą \textbf{pasywnego dochodu}.

\subsubsection{Podstawowe definicje}

\textbf{Giełda} to sesje handlowe realizowane w ustalonym, powszechnie znanym miejscu i czasie. W trakcie sesji dochodzi do umów kupna-sprzedaży, po cenach ogłoszonych w codziennych notowaniach.

\noindent \textbf{Indeks giełdowy} to wartość obliczona na podstawie wyceny akcji wybranych spółek giełdowych. Umożliwia syntetyczne przedstawienie koniunktury na giełdzie lub stanu jakiegoś sektora spółek.

\noindent \textbf{Dochód pasywny} oznacza uzyskiwanie dochodów, w które nie musisz cały czas angażować własnej pracy. Możesz otrzymywać go wtedy, gdy raz wykonana przez Ciebie praca, przynosi zyski przez dłuższy czas. To inaczej aktywa, które generują zyski.

\subsubsection{Gdzie inwestować?}
Do najpopularniejszych platform inwestycyjnych w polsce zaliczają się:

\begin{itemize}
    \item[--] XTB
    \item[--] eTORO
    \item[--] plus 500
\end{itemize}

\subsubsection{Indeksy giełdowe}
Do najpopularniejszych indeksów giełdowych na świecie zaliczają się: 

\begin{enumerate}
    \item S\&P 500
    \item NASDAQ
    \item DJIA
    \item DAX
    \item CAC 40
\end{enumerate}

\subsection{Wykres indeksu S\&P500}

\begin{figure}[h]
    \centering
    \includegraphics[scale=0.3]{pictures/SPX_2021-10-25_14-16-00.png}
    \caption{Wykres S\&P 500}
    \label{fig:wykres500}
\end{figure}

\newpage

\subsection{Tabela zmienności wybranych indeksów}

\begin{table}[h]
    \begin{tabular}{|
    >{\columncolor[HTML]{DAE8FC}}c |l|l|l|l|}
    \hline
    \cellcolor[HTML]{C0C0C0} & \cellcolor[HTML]{DAE8FC}Cena: 26.10.2020 & \cellcolor[HTML]{DAE8FC}Cena: 26.10.2021 & \cellcolor[HTML]{DAE8FC}Zmiana & \cellcolor[HTML]{DAE8FC}Trend \\ \hline
    \multicolumn{1}{|l|}{\cellcolor[HTML]{DAE8FC}S\&P 500} & 3356,44 & 4544,90 & 21\% & $\nearrow$ \\ \hline
    \multicolumn{1}{|l|}{\cellcolor[HTML]{DAE8FC}NASDAQ} & 11642,57 & 15090,20 & 17,08\% & $\nearrow$ \\ \hline
    DIJA & 27595,81 & 35677,02 & 16,57\% & $\nearrow$ \\ \hline
    DAX & 12132,03 & 15585,67 & 28\% & $\nearrow$ \\ \hline
    CAC 40 & 4835,76 & 6733,69 & 38\% & $\nearrow$ \\ \hline
    \end{tabular}
    \caption{Ceny wybranych indeksów year-to-date}
    \label{fig:tab}
\end{table}
\subsection{Obliczenie kontraktu forward na indeksy}
Niech $\delta$ oznacza rynkową (wolną od ryzyka) stopę procentową, zaś przez ciągłą stopę dywidend akcji w indeksie. Wtedy cena kontraktu forward na indeks (zdywidendą) ma postać
$$f(0,T)=S_{0}e^{(\delta - \delta_{q})T}$$
\noindent \textbf{Podstawowe oznaczenia dla opcji:}
\begin{itemize}
    \item $S_{t}$ - proces ceny instrumentu podstawowego dla opcji
    \item $K$ - cena realizacji
    \item $T$ - termin wygaśnięcia opcji
\end{itemize}

\subsection{Podsumowanie}
Mam nadzieję, że po przeczytaniu tego artykułu twoje podejście odnośnie inwestowania zmieni się, może zagłębisz się w ten temat i zdobytą wiedzę przekształcisz w dochód. Powodzenia!
\footnote{Pragnę zaznaczyć, że informacje zawarte w artykule nie są poradą handlową/inwestycyjną i mają jedynie charakter informacyjny}

