\label{sec:Maciej Bobrek}
\newpage
\section{Kryptowaluty}
\subsection{Co to jest kryptowaluta?}
Jest to system przechowujący informacje o systemie posiadania w umownych jednostkach. Informacje te są ściśle związane z tzw. ,,portfelami".  
\subsubsection{Czemu Kryptowaluty są coraz bardziej popularne?}
Główna zaletą kryptowalut jest ich decentralizacja, czyli nie podleganie pod żaden organ dystrybucyjny lub zarządzający daną walutą. Kolejną zaletą jest bezpieczeństwo i anonimowość przy dokonywaniu transkacji, chociaż to drugie może być czasem problemem :).
\subsubsection{Proof of work a Proof of Stake}
Dwa główne sposoby zdobywania kryptowalut to PoW i PoS. PoW wykorzystuje ogromne zasoby enerergii i polega on na udostepnianiu swojej przestrzeni dyskowej lub graficznej w zamiast za daną jednostkę kryptowaluty. Natomiast PoS działa na zasadzie lokaty. Im więcej kryptowaluty posiada ,,miner", tym więcej jej dostaje. W odróżnieniu do Pow nie potrzeba praktycznie w ogóle energii do uzyskania kryptowaluty.
\subsection{Najlopularniejsze Kryptowaluty}
\begin{center}
    \begin{tabular}{|
>{\columncolor[HTML]{FFFFFF}}c |
>{\columncolor[HTML]{FFFFFF}}c |
>{\columncolor[HTML]{FFFFFF}}c |
>{\columncolor[HTML]{FFFFFF}}c |}
\hline
\multicolumn{1}{|l|}{\cellcolor[HTML]{FFFFFF}{\color[HTML]{000000} \textbf{Kryptowaluta}}} & \multicolumn{1}{l|}{\cellcolor[HTML]{FFFFFF}{\color[HTML]{000000} \textbf{Wartość w USD}}} & \multicolumn{1}{l|}{\cellcolor[HTML]{FFFFFF}{\color[HTML]{000000} \textbf{Kapitalizacja Rynkowa w USD}}} & \multicolumn{1}{l|}{\cellcolor[HTML]{FFFFFF}{\color[HTML]{000000} \textbf{Pow czy PoS?}}} \\ \hline
{\color[HTML]{000000} Bitcoin}                                                             & {\color[HTML]{000000} \$61,742.51}                                                         & {\color[HTML]{000000} 1,116 Tryliona}                                                                    & {\color[HTML]{000000} PoW}                                                                \\ \hline
{\color[HTML]{000000} Ethereum}                                                            & {\color[HTML]{000000} \$4,511.49}                                                          & {\color[HTML]{000000} 533 Miliardy}                                                                      & {\color[HTML]{000000} PoW}                                                                \\ \hline
{\color[HTML]{000000} Binance Coin}                                                        & {\color[HTML]{000000} \$605.06}                                                            & {\color[HTML]{000000} 100 Miliardów}                                                                     & {\color[HTML]{000000} PoS}                                                                \\ \hline
\end{tabular}
\end{center}

\subsection{Twierdzenie Pitagorasa}
\begin{equation} \label{pitagoras} a^2+b^2=c^2 \end{equation}

\subsection{Rankingi}
\subsubsection{Giełdy/Portfele Kryptowalut:}
\begin{itemize}
    \item Binance
    \item Etoro
    \item Coinbase
    \item Bitbay
\end{itemize}
\newpage
\subsubsection{Top 5 najlepiej prosperująch kryptowalut według mnie}
\begin{enumerate}
    \item Cardano (ADA)
    \item Polkadot (DOT)
    \item Solana (SOL)
    \item Ethereum (ETH)
    \item Binance Coin (BNB)
\end{enumerate}
\subsection{Wykres BTC do USD (ostatnie 5 lat}

\begin{figure}[h]
    \centering
    \includegraphics[scale=1.2]{pictures/wykres.png}
    \caption{wykres liniowy BTC/USD}
    \label{fig:wykresBTC}
\end{figure}
